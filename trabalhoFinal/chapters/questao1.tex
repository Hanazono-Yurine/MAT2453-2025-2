\documentclass[../resumo.tex]{subfiles}
\graphicspath{{\subfix{../images/}}}

\begin{document}

	\subsection{Enunciado}

	Cálculo da integral gaussiana, $\int_{-\infty}^{\infty} e^{-x^2} dx$, utilizando o chamado 
	“Feynman’s trick” assim trabalhar	com um parâmetro real (auxiliar) de modo a considerar uma
	integral dependente desse parâmetro, digamos $f$, $\int_{-\infty}^{\infty} F(x, t) dx = \varphi(t)$
	que a integral gaussiana é um dos valores de $varphi$ digamos para $t = 0$,
	$\varphi(0) = \int_{-\infty}^{\infty} F(x,0) dx$ e $F(x,0) = e^{-x^2}$, além disso o “truque de Feynman”
	entra em cena ao se considerar $\varphi'(t)$ e iguala-lá à
	$\int_{-\infty}^{\infty} \dfrac{\partial F}{\partial t}(x,t) dx$,	relaciona - lá novamente com $\varphi(t)$,
	obtendo uma equação diferencial relativamente simples para $\varphi(t)$, determinar $\varphi$ a partir disso,
	e finalmente calcular $\varphi(0)$.

	\subsection{Escolha de $F(x,t)$ (condição $F(x,t) = e^{-x^2}$)}

	Escolhemos a forma simples que preserva a estrutura gaussiana:

	\[ F(x,t) = e^{-(1+t)x^2} \]

	Verifique: $F(x,0) = e^{-x^2}$, como exigido. Para convergência das integrais precisamos $1+t > 0$,
	então trabalhamos para $t > -1$.

	Defina

	\begin{align*}
		\varphi(t) &= \int_{-\infty}^{\infty} F(x,t) dx \\
							 &= \int_{-\infty}^{\infty} e^{-(1+t)x^2} dx, \quad t > -1
	\end{align*}

	\subsection{Justificativa para troca derivada e integral}

	Queremos calcular $\varphi'(t)$ por

	\begin{align*}
		\varphi'(t) &= \dfrac{d}{dt} \int_{-\infty}^{\infty} F(x,t) dx \\
								&= \int_{-\infty}^{\infty} \dfrac{\partial F}{\partial t}(x,t) dx
	\end{align*}

	Para trocar derivada e integral usamos o \textbf{Teorema da Convergência Dominada}.


	Calculamos a derivada parcial:

	\begin{align*}
		\frac{\partial F}{\partial t}(x,t) &= \frac{\partial}{\partial t} e^{-(1+t)x^2} \\
																			 &= -x^2 e^{-(1+t)x^2}
	\end{align*}

	Se fixarmos um intervalo compacto de $t$ contido em $[-1, \infty]$, digamos $t \in [t_0, t_1]$ com $t_0 > -1$,
	então $1 + t \geq 1 + t_0 > 0$ nesse intervalo. Logo

	\[ \left[ -x^2 e^{-(1+t)x^2} \right] \leq x^2 e^{-(1+t_0)x^2} \]

	A função $x \longmapsto x^2 e^{-(1+t_0)x^2}$ e integrável em $\mathbb{R}$ (uma gaussiana domina qualquer polinômio).
	Assim existe uma função integrável que domina $| \varphi, F (x,t)|$ uniformemente em $t \in [t_0, t_1]$.

	Portanto, por convergência dominada, a troca é válida e podemos escrever

	\[ \varphi'(t) = \int_{-\infty}^{\infty} \frac{\partial F}{\partial t}(x,t) dx \]

	\subsection{Cálculo de $\varphi'(t)$}

	\begin{align}
		\varphi'(t) &= \int_{-\infty}^{\infty} e^{-(1+t)x^2} dx \notag\\
								&= -\int_{-\infty}^{\infty} x^2 e^{-(1+t)x^2}dx \label{eq:a}
	\end{align}

	\subsection{Relacionar essa integral com $\varphi(t)$ (mudança de variável)}

	Trocar $\varphi(t)$ por uma variável conveniente.

	Comece por escrever $\varphi(t)$:

	\[ \varphi(t) = \int_{-\infty}^{\infty} e^{-(1+t)x^2} dr \]

	Faça $u = \sqrt{1+t} \; x$. Então

	\begin{align*}
		x &= \frac{u}{\sqrt{1+t}} \\
			&= \frac{1}{\sqrt{1+t}} \int_{-\infty}^{\infty} e^{-u^2} du
	\end{align*}

	Defina a constante (independente de $t$)

	\[ C = \int_{-\infty}^{\infty} e^{-u^2} du \]

	Logo

	\begin{align}
		\varphi(t) = C(1+t)^{-\dfrac{1}{2}} \label{eq:b}
	\end{align}

	Em \eqref{eq:a} substitua $x = \dfrac{u}{\sqrt{1+t}}$. Então $x^2 = \dfrac{u^2}{1+t}$ e
	$dx = \dfrac{du}{\sqrt{1+t}}$. Assim

	\begin{align*}
		&\quad \int_{-\infty}^{\infty} x^2 e^{-(1+t)x^2} dx \\
		&= \int_{-\infty}^{\infty} \frac{u^2}{1+t} e^{-u^2} \frac{du}{\sqrt{1+t}} \\
		&= \frac{1}{(1+t)^{\dfrac{3}{2}}} \int_{-\infty}^{\infty} u^2 e^{-u^2} du
	\end{align*}

	Portanto \eqref{eq:a} vira

	\begin{align}
		\varphi'(t) = -\frac{1}{(1+t)^{\dfrac{3}{2}}} \int_{-\infty}^{\infty} u^2 e^{-u^2} du \label{eq:c}
	\end{align}

	Podemos agora relacionar \eqref{eq:c} com \eqref{eq:b} derivando \eqref{eq:b} (é mais direto):

	Derivando \eqref{eq:b}

	\begin{align*}
		\varphi'(t) &= C \cdot \left( -\frac{1}{2} \right)(1+t)^{-\dfrac{3}{2}} \\
								&= - \frac{1}{2} C (1+t)^{-\dfrac{3}{2}}
	\end{align*}

	Compare isso com \eqref{eq:c}. Para compatibilidade, deve valer

	\begin{align*}
		&\quad \int_{-\infty}^{\infty} u^2 e^{-u^2} du \\
		&= \frac{1}{2} \int_{-\infty}^{\infty} e^{-u^2} du \\
		&= \frac{1}{2} C
	\end{align*}

	o que é verdade (pode-se ver por integração por partes ou usando simetrias).
	Assim obtemos a EDO diretamente:

	\begin{align}
		\varphi'(t) = -\frac{1}{2(1+t)} \varphi(t) \label{eq:d}
	\end{align}

	\subsection{Resolver a EDO detalhadamente (separação de variáveis)}

	A EDO \eqref{eq:d} é de primeira ordem e separável:


	\[ \frac{d\varphi}{dt} = -\frac{1}{2(1+t)} \; \varphi(t) \]

	Separe as variáveis

	\[ \frac{1}{\varphi} d \varphi = -\frac{1}{2(1+t)} dt \]

	Integre ambos os lados (integral indefinida):

	Integral $\dfrac{1}{\varphi}$ dp = integral $-\dfrac{1}{2(1+t}dt) \Rightarrow \ln|\varphi| = \dfrac{1}{2}\ln|1+t| + C_1$

	onde $C_1$ é uma constante de integração.

	Exponenciando,

	\[ |\varphi| = e^{C_1} |1+t| x^{-\dfrac{1}{2}} \]

	Com $\varphi(t) > 0$ para $t > -1$, podemos escrever sem módulo:

	\[ \varphi(t) = C (1+t)^{-\dfrac{1}{2}}, \quad t > -1 \]

	\subsection{Determinar a constante $C=\varphi(0)$ (método do quadrado)}

	Para achar

	\[ C = \varphi(0) = \int_{-\infty}^{\infty} e^{-x^2} dx \]

	Denote $I = \int_{-\infty}^{\infty} e^{-x^2} dx$

	Escreva $I^2$ como produto de duas integrais idênticas:

	\[ I^2 = \left( \int_{-\infty}^{\infty} e^{-x^2} dx \right) \left( \int_{-\infty}^{\infty} e^{-y^2} dy \right) \]

	Combine em integral dupla:

	\[ I^2 = \iint_{\mathbb{R}}^{} e^{-|x^2 + y^2|} dxdy \]

	Mude para coordenadas polares: $x = r \cos \theta, y = r\sin \theta$

	Jacobiano: $dx dy = r \; dr \; d \; \theta $

	Região $\mathbb{R}$ corresponde a $r \in [0, \infty], \theta \in [0, 2\pi]$. Assim:

	\[ I^2 = \int_{0}^{2\pi} \int_{0}^{\infty} e^{-r^2 r \; dr \; d \; \theta} \]

	Separe integrais (produto):

	\begin{align*}
		I^2 &= \left( \int_{0}^{2\pi} d \theta \right) \left( \int_{0}^{\infty} r \; e^{-r^2} \; dr \theta \right) \\
				&= 2 \pi \int_{0}^{\infty} r \; e^{-r^2} \; dr
	\end{align*}

	Calcule $\int_{0}^{\infty} r \; e^{-r^2} \; dr$ po substituição.

	Seja $u = r^2$. Então $du = 2r \Rightarrow r \; dr = \frac{1}{2} du$.

	Quando $r:0 \rightarrow \infty$, então $u:0 \rightarrow \infty$.

	Portanto

	\[\int_{0}^{\infty} r \; e^{-r^2} dr = \frac{1}{2} \int_{0}^{\infty} e^{-u} du \]

	E

	\[ \int_{0}^{\infty} e^{-u} du = \left[ -e^{-u} \right]_{0}^{\infty} = 1 \]

	Logo

	\[ \int_{0}^{\infty} r \; e^{-r^2} dr = \frac{1}{2} \]

	Substitua de volta:

	\[ I^2 = 2 \pi \frac{1}{2} = \pi \]

	Como $I > 0$ conclui-se:

	\[ I = \sqrt{\pi} \]

	\subsection{Resultado e verificação}

	Substituindo $C = \sqrt{\pi}$ em $\varphi(t) = C(1+t)^{-\dfrac{1}{2}}$ obtemos, para $t > -1$

	\begin{align*}
		\varphi(t) &= \int_{-\infty}^{\infty} F(x,t) dx \\
							 &= \int_{-\infty}^{\infty} e^{(-1+t)x^2} dx \\
							 &= \frac{\sqrt{\pi}}{\sqrt{1+t}}
	\end{align*}

	\begin{align*}
		\varphi(0) &= \int_{-\infty}^{\infty} F(x,0) dx \\
							 &= \int_{-\infty}^{\infty} e^{-x^2} dx \\
							 &= \sqrt{\pi}
	\end{align*}
	

\end{document}
