\documentclass[../resumo.tex]{subfiles}
\graphicspath{{\subfix{../images/}}}

\begin{document}

	\subsection{Enunciado}

	Verifique que a função f(x) descrita abaixo é derivável, porém sua derivada
	$f'$ não é contínua em $x = 0$

	\begin{equation*}
			f(x) =
			\begin{cases}
					x^2\sin{\frac{1}{x}}, \quad x \neq 0 \\
					0, \quad x = 0
			\end{cases}
	\end{equation*}

	\subsection{Continuidade em x = 0}

	Para que essa função seja contínua em $x = 0$, temos que:
	\[\lim_{x\to0} x\cdot\sin{\frac{1}{x}} = 0\]

	Novamente, será utilizada a definição de limite por épsilon-delta para provar a continuidade.

	Quer se provar que para todo $\varepsilon > 0$, existe um $\delta > 0$ tal que, se
	$0 < |x - 0| < \delta$, então $|f(x) - 0| < \varepsilon$.
	\begin{enumerate}
			\item Seja $\varepsilon > 0$ um número real positivo qualquer
			\item Precisamos encontrar um $\delta > 0$ correspondente. Analisando a expressão $|f(x) - 0|$, temos:
			\[|f(x) - 0| = x\cdot\sin{\frac{1}{x}} = |x|\cdot\sin{\frac{1}{x}}\]
			\item Sabemos que a função seno é limitada, uma vez que, para qualquer valor de $\theta$, temos que
			$\sin{theta} \leq 1$. Portanto:
			\[\sin{\frac{1}{x}} \leq 1\]
			\item Substituindo essa desigualdde na expressão, obtemos:
			\[|x|\cdot\sin{\frac{1}{x}} \leq |x|\cdot1 = |x|\]
			\item Já foi mostrado que $|f(x) - 0| \leq |x|$. Portanto, se também garantirmos que
			$|x| \leq \varepsilon$, obteremos nossa prova.
			\item Podemos então escolher $\delta = \varepsilon$. Assim, assumindo que $0 < |x| < \delta$, temos
			que $|x| < \varepsilon$
			E como $|f(x)-0| \leq |x|$, concluimos que:
			\[|f(x)-0 < \varepsilon\]
	\end{enumerate}

	Portanto, o limite existe e é 0. Como $\lim_{x\to0}f(x) = 0$ e $f(0) = 0$, conclui-se que a função
	é contínua em $x = 0$.

	\subsection{Derivabilidade em x = 0}

	Para verificar se a função é derivável em $x = 0$, basta calcular o valor de sua derivada, a partir
	da definição de limite:
	\[f'(0) = \lim_{h\to0} \frac{f(0 + h) - f(0)}{h}\]

	Substituindo os valores na função, a expressão se torna:

	\begin{align*}
			f'(0) &= \lim_{h\to0} \frac{h\cdot\sin{\frac{1}{h}} - 0}{h} \\
			f'(0) &= \lim_{h\to0} \frac{h\cdot\sin{\frac{1}{h}}}{h} \\
			f'(0) &= \lim_{h\to0}\sin{\frac{1}{h}}
	\end{align*}

	De forma análoga ao item anterior, novamente há a ocorrência do limite da forma $\lim_{x\to0} \sin{1}{x}$,
	que não existe. Dessa forma, conclui-se que a função não é derivável em $x = 0$.

	\subsection{Gráfico da função}

	\begin{figure}[H]
			\centering
			\includegraphics[width=0.5\textwidth]{images/ex22}
			\caption{Gráfico de f(x)}
	\end{figure}


\end{document}
