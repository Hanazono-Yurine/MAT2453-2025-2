\documentclass[../resumo.tex]{subfiles}
\graphicspath{{\subfix{../images/}}}

\begin{document}

	\subsection{Enunciado}

	Considerando a função $\varphi(x) = x - 2x^3$, $x \in \mathbb{R} (\mathbb{R} \rightarrow \mathbb{R})$
	determinar (e argumentar, justificando) para quais $y \in \mathbb{R}$ a função $\varphi(x) = y$ (na
	incógnita $x$) tem 1 solução, 2 soluções, 3 soluções.

	\subsection{Estudo da função $\varphi(x) = x - 2x^3$}
	
	A análise é feita através do estudo da derivada primeira e segunda, da identificação dos
	extremos locais e do comportamento global da função cúbica.

	\subsubsection{Definição da função}

	Seja $\varphi(x) = x - 2x^3$, com $\varphi: \mathbb{R} \rightarrow \mathbb{R}$.

	\subsubsection{Derivada primeira e pontos críticos}

	A derivada primeira é:

	\[ \varphi'(x) = \frac{d}{dx}(x - 2x^3) = 1 - 6x^2 \]
	
	Para encontrar os pontos críticos, impõe-se $\varphi'(x) = 0$:
	
	\[ 1 - 6x^2 = 0 \rightarrow x = \frac{\pm 1}{\sqrt{6}} \]

	\subsubsection{Derivada segunda e classificação dos extremos}

	A derivada segunda é:

	\[ \varphi''(x) = \frac{d}{dx}(1 - 6x^2) = -12x \]

	Cálculo da segunda derivada nos pontos críticos:

	\begin{itemize}
		\item Ponto de máximo local: $\varphi''\frac{1}{\sqrt{6}} = -12\frac{1}{\sqrt{6}} = -2\sqrt{6}$
		\item Ponto de mínimo local: $\varphi''\frac{-1}{\sqrt{6}} = -12\frac{-1}{\sqrt{6}} = 2\sqrt{6}$
	\end{itemize}

	\subsubsection{Cálculo dos valores máximos e mínimos locais}

	\begin{align*}
		\varphi(\frac{1}{\sqrt{6}}) &= \frac{1}{\sqrt{6}} - 2(\frac{1}{\sqrt{6}})^3 \\
																&= \frac{1}{\sqrt{6}} - \frac{2}{6\sqrt{6}} \\
																&= \frac{2}{3}\frac{1}{\sqrt{6}} \\
																&= \frac{2}{(3\sqrt{6})} \\
																&= \frac{\sqrt{6}}{9}
	\end{align*}

	\begin{align*}
		\varphi(\frac{-1}{\sqrt{6}}) &= \frac{-1}{\sqrt{6}} - 2(\frac{-1}{\sqrt{6}})^3 \\
																 &= \frac{-1}{\sqrt{6}} + \frac{2}{6\sqrt{6}} \\
																 &= \frac{-2}{3}\frac{1}{\sqrt{6}} \\
																 &= \frac{-\sqrt{6}}{9}
	\end{align*}

	Logo, o máximo local é $y_{max} = \frac{\sqrt{6}}{9}$ e o mínimo local é $y_{min} = \frac{-\sqrt{6}}{9}$.

	\subsubsection{Comportamento da função nos extremos do domínio}

	Para $x \rightarrow +\infty$, o termo dominante é $-2x^3$, logo $\varphi(x) \rightarrow -\infty$.
	Para $x \rightarrow -\infty$, $\varphi(x) \rightarrow +\infty$.

	Como $\varphi(x)$ é um polinômio de grau ímpar e contínuo, o conjunto imagem é $\mathbb{R}$, garantindo
	pelo menos uma solução real para qualquer $y \in \mathbb{R}$.

	\subsubsection{Determinação do número de soluções para diferentes valores de y}

	Se $y > \frac{\sqrt{6}}{9} \rightarrow 1$ solução real
	(reta horizontal acima do máximo local intercepta o gráfico uma única vez).

	Se $y = \frac{\sqrt{6}}{9} \rightarrow 2$ soluções reais
	(uma é o ponto de tangência $x = \frac{1}{\sqrt{6}}$, a outra é uma interseção simples).

	Se $\frac{-\sqrt{6}}{9} < y < \sqrt{6}{9} \rightarrow 3$ soluções reais
	(reta corta o gráfico nas três regiões de monotonicidade).

	Se $y = \frac{-\sqrt{6}}{9} \rightarrow 2$ soluções reais
	(uma é o ponto de tangência $x = \frac{-1}{\sqrt{6}}$, a outra é uma interseção simples).

	Se $y < \frac{-\sqrt{6}}{9} \rightarrow 1$ solução real
	(reta abaixo do mínimo local intercepta o gráfico uma única vez).

	\subsubsection{Multiplicidade das raízes}

	Para $y = \frac{\pm \sqrt{6}}{9}$, a raiz correspondente ao ponto crítico é de multiplicidade 2 (tangência).
	Nos demais casos, todas as raízes são simples.

	\subsubsection{Conclusão}

	A equação $x - 2x^3 = y$ apresenta:
	\begin{enumerate}
		\item 1 solução real se $y > \frac{\sqrt{6}}{9}$ ou $y < \frac{-\sqrt{6}}{9}$
		\item 2 soluções reais se $y = \frac{\pm \sqrt{6}}{9}$
		\item 3 soluções reais se $\frac{-\sqrt{6}}{9} < y < \frac{\sqrt{6}}{9}$
	\end{enumerate}

	\subsubsection{Gráficos}

	\begin{figure}[H]
			\centering
			\includegraphics[width=0.5\textwidth]{images/ex5-original}
			\caption{Gráfico de $\varphi(x) = x - 2x^3$}
	\end{figure}

	\begin{figure}[H]
			\centering
			\includegraphics[width=0.5\textwidth]{images/ex5-primeira}
			\caption{Gráfico da derivada primeira de $\varphi(x) = x - 2x^3$}
	\end{figure}

	\begin{figure}[H]
			\centering
			\includegraphics[width=0.5\textwidth]{images/ex5-segunda}
			\caption{Gráfico da derivada segunda de $\varphi(x) = x - 2x^3$}
	\end{figure}

\end{document}
