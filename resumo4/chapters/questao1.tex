\documentclass[../resumo.tex]{subfiles}
\graphicspath{{\subfix{../images/}}}

\begin{document}

	\subsection{Enunciado}

	Verifique que a função f(x), descrita abaixo, é contínua, porém não derivável em $x = 0$.

	\begin{equation*}
			f(x) = 
			\begin{cases}
					x\sin{x}, \quad x \neq 0 \\
					0, \quad x = 0
			\end{cases}
	\end{equation*}

	\subsection{Continuidade em x = 0}

	Para verificar a continuidade em $x = 0$, é necessário provar que $\lim_{x\to0}f(x) = f(0)$, ou seja, 
	é necessário mostrar que:

	\begin{equation*}
			\lim_{x\to0}x\cdot\sin{\frac{1}{x}} = 0
	\end{equation*}

	% segunda parte - prova 

	\subsection{Prova pela Definição (Épsilon-Delta)}

	Quer se provar que para todo $\varepsilon > 0$, existe um $\delta > 0$ tal que, se $0 < |x-0| < \delta$, então:
	$|f(x) - o < \varepsilon$

	\begin{enumerate}
			\item Seja $\varepsilon > 0$ um número real positivo qualquer
			\item Precisamos encontrar um $\delta > 0$ correspondente. Para isso, vamos analisar
			a expressão $|f(x) - 0|$.
			\[|f(x) - 0| = |x|\cdot\sin{\frac{1}{x}}\] 
			\item Sabe-se que a função seno é limitada, uma vez que, para qualquer
			valor de $\theta$, temos que $|\sin{\theta}| \leq 1$, portanto, $\sin{\frac{1}{x} \leq 1}$
			\item substituindo essa desigualdade na nossa expressão, obtemos
			\[|x|\cdot\sin{\frac{1}{x}} \leq |x|\cdot1 = |x|\]
			\item O nosso objetico é fazer com que $|f(x) - 0 < \varepsilon$. Já foi mostrado que
			$|f(x) - 0| \leq |x|$. Para isso, basta garantir que $|x| < \varepsilon$.
			\item Pode-se então escolher $\delta = \varepsilon$. Assim, se assumirmos que 
			$0 < |x| < \delta$, teremos $|x| < \varepsilon$. \\
			E como $|f(x) - 0| \leq |x|$, concluímos que:

			\[|f(x) - 0| < \varepsilon\]
	\end{enumerate}

	Portanto, o limite existe e é 0. Como $\lim_{x\to0} = 0$ e $f(0) = 0$, a função é contínua em $x = 0$.


	%-------------------------------------------------------------

	\subsection{Derivabilidade em x = 0}

	Para verificar se a função é derivável em $x = 0$, precisamos calcular sua derivada a partir da definição
	de limite:
	\[f'(0) = \lim_{h\to0} \frac{f(0 + h) - f(0)}{h}\]

	Substituindo os valores, temos que:

	\begin{align*}
			f'(0) &= \lim_{h\to0} \frac{h\cdot\sin{\frac{1}{h}} - 0}{h} \\
			f'(0) &= \lim_{h\to0} \frac{h\cdot\sin{\frac{1}{h}}}{h} \\
			f'(0) &= \lim_{h\to0} \sin{\frac{1}{h}}
	\end{align*}

	Note que, na última relação, chegamos a um limite da forma $\sin{\frac{1}{x}}$, com x tendendo a 0.
	Esse limite não existe, e em decorrência disso, conclui-se que a função do enunciado não é derivável em
	 $x = 0$.

 % -------------------------

 \subsection{Gráfico da função}

 \begin{figure}[H]
    \centering
    \includegraphics[width=0.5\textwidth]{images/ex11}
    \caption{Gráfico da função f(x)}
 \end{figure}


\end{document}
