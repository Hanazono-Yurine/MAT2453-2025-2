\documentclass[../resumo.tex]{subfiles}
\graphicspath{{\subfix{../images/}}}

\begin{document}

	Com a equação quadrática:

	\[ ax^2 + bx + c = 0, (a = 0) \]

	É possível obter o $\Delta$, ou discriminate, que é o número que decide a natureza das raízes da equação quadrática.
	O $\Delta$ vem da fórmula de bhaskara, obtida pelo método de completar quadrados 

	\[ ax^2 + bx + c = 0 \]

	\textit{(assumindo $a \neq 0$):}

	\begin{align*}
		x^2 + (\frac{b}{a})x + (\frac{c}{a}) &= 0 \\
		x^2 + (\frac{b}{a})x &= \frac{-c}{a}
	\end{align*}

	Adicionando $(\frac{b}{2a})^2$ a ambos os lados da equação para transformar o lado esquerdo em um trinômio quadrado perfeito.

	\begin{align*}
		x^2 + (\frac{b}{a})x + (\frac{b}{2a})^2 &= -\frac{c}{a} + (\frac{b}{2a})^2 \\
		x^2 + (\frac{b}{a})x + \frac{b^2}{4a^2} &= -\frac{c}{a} + \frac{b^2}{4a^2} \\
		(x + \frac{b}{2a})^2 &= \frac{b^2}{4a^2} - \frac{c}{a} \\
		(x + \frac{b}{2a})^2 &= \frac{(b^2 - 4ac)}{4a^2} \\
		x + \frac{b}{2a} &= \pm\frac{\surd(b^2 - 4ac)}{\surd(4a^2)} \\
		x + \frac{b}{2a} &= \pm\frac{\surd(b^2 - 4ac)}{2a} \\
		x &= \frac{-b}{2a} \pm \frac{\surd(b^2 - 4ac)}{2a} \\
		x &= \frac{-b \pm \surd(b^2 - 4ac)}{2a} \\
		\Delta &= b^2 - 4ac \\
	\end{align*}

	\subsection{Interpretações do Delta}

	\subsubsection{Natureza e quantidade de raízes reais}
	
	\begin{itemize}
		\item $\Delta > 0 \to$ duas raízes reais \textbf{distintas}.
		\item $\Delta = 0 \rightarrow$ \textbf{uma} raiz real \textbf{dupla} (as duas coincidem).
		\item $\Delta < 0 \rightarrow$ \textbf{nenhuma} raiz real (raízes complexas conjugadas).
	\end{itemize}

	Equivalentemente, no gráfico de :
	
	\begin{itemize}
		\item $\Delta > 0 \rightarrow$ a parábola corta o eixo em \textbf{dois} pontos.
		\item $\Delta = 0 \rightarrow$ a parábola \textbf{toca} o eixo no \textbf{vértice}.
		\item $\Delta < 0 \rightarrow$ a parábola \textbf{não} intercepta o eixo .
	\end{itemize}

	Se conhecemos as raízes, conseguimos fatorar o polinômio.

	\subsubsection{Caso 1: $\Delta > 0$ (duas raízes reais)}

	Se  são as raízes:

	\[ ax^2 + bx + c = a(x - x_1)(x - x_2) \]

	\subsubsection{Caso 2: $\Delta = 0$ (raiz dupla)}

	Se a única raiz é :

	\[ ax^2 + bx + c = a(x - x_0)^2 \]

	\subsubsection{Caso 3: $\Delta < 0$ (raízes complexas)}

	Não dá para fatorar em $R$ . Mas em $C$ :

	\[ ax^2 + bx + c = a(x - z_1)(x - z_2) \]

\end{document}
