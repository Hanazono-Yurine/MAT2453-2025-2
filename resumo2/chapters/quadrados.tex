\documentclass[../resumo.tex]{subfiles}
\graphicspath{{\subfix{../images/}}}

\begin{document}

	Uma das formas de se encontrar as raizes de uma função é o processo de completar quadrados.

	O grau de um equação está relacionado a quantidade de soluções de uma função, ou seja, uma função
	de primeiro grau terá uma única solução, um função de segundo grau terá duas soluções e assim por diante.

	Funções de segundo grau normalmente se encontram na seguinte forma:

	\[ ax^2 + bx + c = 0 \]

	Existe diferentes formas de se resolver uma equação utilizando o metdo de completar quadrados, tudo depende 
	da própria equação.

	\subsection{É Trinômio Quadrado Perfeito}

	[WIP] Esplicar oque é um trinômio perfeito

	Para compreender esse conceito, é necessário abordá-lo 


	Quando a equação é um trinômio quadrado perfeito, temos que primeiro tranforma-la em um produto notável
	e depois resolve-lo.

	Pegando como exemplo a seguinte equação:

	\[ x^2 + 6x + 9 = 0 \]

	Temos os coeficientes $b = 6$ e $c = 9$.

	Assim podemos escrever a equação como

	\[ x^2 + 6x + 9 = (x + 3)^2 = 0 \]

	e já que um produto notável nada mais é que o produto de dois polinômios iguais, podemos transformar 
	essa equação em

	\[ (x + 3)^2 = (x + 3)(x + 3) = 0 \]

	E para um produto ser igual a zero, um de seus fatores tem que ser igual a zero. Portanto no nosso exemplo
	é necessário que $(x + 3) = 0$ ou que $(x + 3) = 0$.

	Resolvendo, temos que para ambos $(x + 3) = 0$, $x$ será $-3$. 

	\subsection{Não é Trinômio Quadrado Perfeito}

	\subsection{Coeficiente "a" diferente de 1}

\end{document}
