\documentclass[../resumo.tex]{subfiles}
\graphicspath{{\subfix{../images/}}}

\begin{document}

	Uma das formas de se encontrar as raizes de uma função é o processo de completar quadrados.

	O grau de uma equação está relacionado a quantidade de soluções de uma função - uma função
	de primeiro grau terá uma única solução, uma função de segundo grau terá duas soluções e assim por diante.

	Funções de segundo grau normalmente são representadas na seguinte forma:

	\[ ax^2 + bx + c = 0 \]

	Existe diferentes formas de se resolver uma equação utilizando o método de completar quadrados
	mas, de modo geral, todas elas envolvem algum dos produtos notáveis abaixo:

	\subsection{Trinômio Quadrado Perfeito}

	Para compreender plenamente esse conceito, é necessário primeiro antes definir o conceito de polinômios.

	\subsubsection{Polinômios - conceitos básicos}

	Polinômios são expressões matemáticas da forma:
	\[a_1x^n + a_2x^{n-1} + ... + a_{n-1}x^1 + a_n \quad \textrm{t.q.} \quad a \in \mathbb{R}, n \in \mathbb{Z}\]

	Note que cada termo \(a_nx^n\) é denominado monômio, que apresenta coeficiente \(a\) e grau \(n\).
	Note que o grau do termo \(a_0\) é nulo, e caso o termo seja igual a 0, ele não possui grau.

	Sendo assim, um trinômio é definido como um polinômio de três termos de graus distintos e não nulos. Um exemplo de
	trinômio pode ser visto abaixo:
	\[x^2 - 5x + 6\]

	\subsubsection{Trinômio do Quadrado Perfeito - Definição}

	Agora que sabemos o que é um trinômio, é necessário compreender o que o classifica como "do 
	quadrado perfeito". Um trinômio do quadrado perfeito é aquele que pode ser reduzido à forma:

	\[(a + b)^2\]
	ou;
	\[(a - b)^2\]

	Reescrevendo a primeira expressão como \((a + b)(a + b)\), e realizando uma distributiva, chegamos
	à seguinte relação:

	\[(a + b)^2 = a^2 + 2ab + b^2\]
	

	Visualizando esses termos geométricamente, nota-se que eles resultam em um quadrado,
	de lado \(x + y\), como pode ser visto na figura abaixo. É a partir dessa característica
	que um trinômio é classificado como quadrado perfeito.

	\begin{figure}[H]
		% figura do trinomio do quadrado perfeito
		\centering
		\includegraphics[scale=.5]{trinomio.png}
		\label{fig: Visualização geométrica do trinômio quadrado perfeito}
	\end{figure}

	Perceba que a mesma manipulação também pode ser feita com a expressão \((a - b)^2\), que resultará em:

	\[(a - b)^2 = a^2 - 2ab + b^2\]


	\subsubsection{Resolvendo equações de segundo grau com trinômios quadrados perfeitos}

	A partir da definição dada anteriormente de trinômio quadrado perfeito, pode-se resolver
	algumas equações de segundo grau. Para tal, e necessário que a equação em questão possa ser escrita
	na forma \((a + b)^2\), vejamos por exemplo, a equação:
	
	\[x^2 + 6x + 9\]


	Quando a equação é um trinômio quadrado perfeito, temos que primeiro tranforma-la em um produto notável
	e depois resolve-lo.

	Pegando como exemplo a seguinte equação:

	\[ x^2 + 6x + 9 = 0 \]

	Temos os coeficientes $b = 6$ e $c = 9$.

	Assim podemos escrever a equação como

	\[ x^2 + 6x + 9 = (x + 3)^2 = 0 \]

	e já que um produto notável nada mais é que o produto de dois polinômios iguais, podemos transformar 
	essa equação em

	\[ (x + 3)^2 = (x + 3)(x + 3) = 0 \]

	E para um produto ser igual a zero, um de seus fatores tem que ser igual a zero. Portanto no nosso exemplo
	é necessário que $(x + 3) = 0$ ou que $(x + 3) = 0$.

	Resolvendo, temos que para ambos $(x + 3) = 0$, $x$ será $-3$. 

	\subsection{Não é Trinômio Quadrado Perfeito}

	\subsection{Coeficiente "a" diferente de 1}

	Para as equações cujos coeficientes $a$ sejam diferentes de 1, basta
	realizar operações algébricas de modo a torná-lo 1. O jeito mais simples
	é dividir todos os termos da equação por $a$.

	\[ax^2 + bx + c = 0 \quad \leftrightarrow \]

	\[\frac{ax^2}{a} + \frac{bx}{a} + \frac{c}{a} = 0 \quad \leftrightarrow \]

	\[x^2 + \frac{bx}{a} + \frac{c}{a} = 0\]

	Com essa última equação, o método de completar quadrados pode ser aplicado.

\end{document}
