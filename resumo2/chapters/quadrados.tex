\documentclass[../resumo.tex]{subfiles}
\graphicspath{{\subfix{../images/}}}

\begin{document}

	Uma das formas de se encontrar as raizes de uma equação é o processo de completar quadrados.

	O grau de uma equação está relacionado à quantidade de soluções que apresenta - uma equação
	de primeiro grau terá uma única solução, uma de segundo grau terá duas soluções, e assim por diante.

	Equações de segundo grau normalmente são representadas na seguinte forma:

	\[ ax^2 + bx + c = 0 \]

	Existem diferentes formas de se resolver uma equação de segundo grau. Uma delas é o método de 
	completar quadrados. Esse método consiste na aplicação do conceito de trinômios quadrados
	perfeitos em equações de segundo grau.

	\subsection{Trinômio Quadrado Perfeito}

	Para compreender plenamente esse conceito, é necessário primeiro antes definir o conceito de polinômios.

	\subsubsection{Polinômios - conceitos básicos}

	Polinômios são expressões matemáticas da forma:
	\[a_1x^n + a_2x^{n-1} + ... + a_{n-1}x^1 + a_n \quad \textrm{t.q.} \quad a \in \mathbb{R}, n \in \mathbb{Z}\]

	Note que cada termo \(a_nx^n\) é denominado monômio, que apresenta coeficiente \(a\) e grau \(n\).
	Note que o grau do termo \(a_0\) é nulo, pois ele pode ser escrito na forma \(a_nx^0\).

	Sendo assim, um trinômio é definido como um polinômio de três termos de graus distintos e não nulos. Um exemplo de
	trinômio pode ser visto abaixo:
	\[x^2 - 5x + 6\]

	\subsubsection{Trinômio do Quadrado Perfeito - Definição}

	Agora que sabemos o que é um trinômio, é necessário compreender o que o classifica como "do 
	quadrado perfeito". Um trinômio do quadrado perfeito é um trinômio que pode ser escrito
	como o quadrado de um binômio. Assim sendo, um trinômio quadrado perfeito pode ser descrito por:

	\[(a + b)^2\]
	ou;
	\[(a - b)^2\]

	Reescrevendo a primeira expressão como \((a + b)(a + b)\), e realizando uma distributiva, chegamos
	à seguinte relação:

	\[(a + b)^2 = a^2 + 2ab + b^2\]
	

	Visualizando esses termos geométricamente, nota-se que eles resultam em um quadrado,
	de lado \(a + b\), como pode ser visto na figura abaixo. É a partir dessa característica
	que um trinômio é classificado como quadrado perfeito.

	\begin{figure}[H]
		% figura do trinomio do quadrado perfeito
		\centering
		\includegraphics[scale=.5]{trinomio.png}
		\label{fig: Visualização geométrica do trinômio quadrado perfeito}
	\end{figure}

	Perceba que a mesma manipulação também pode ser feita com a expressão \((a - b)^2\), que resultará em:

	\[(a - b)^2 = a^2 - 2ab + b^2\]


	\subsubsection{Resolvendo equações de segundo grau com trinômios quadrados perfeitos}

	A partir da definição dada anteriormente de trinômio quadrado perfeito, pode-se resolver
	algumas equações de segundo grau. Para tal, é necessário que a equação em questão possa ser escrita
	na forma \((a + b)^2\), vejamos por exemplo, a equação:
	
	\[x^2 + 6x + 9 = 0 \quad \leftrightarrow\]

	\[x^2 + 2\cdot x\cdot3 + 3^2 = 0 \quad \leftrightarrow\]

	\[(x + 3)^2 = 0 \quad \leftrightarrow\]

	\[(x + 3)(x + 3) = 0\]

	Essa última equação, equivalente à primeira apresentada, e apresenta solução trivial, uma vez que, para que
	essa multiplicação seja igual a 0, \(x = -3\)

	Mesmo que inicialmente essa equação pareça apresentar apenas uma solução,
	por ela ser de segundo grau, ela apresenta duas soluções, que nesse caso são iguais. Assim, podemos dizer
	que $-3$ é uma raiz com dupla multiplicade da equação \(x^2 + 6x + 9 = 0\).

	\subsection{Outras expressões}

	Quando nos deparamos com equações de segundo grau que não podem ser expressas diretamente
	na forma de um trinômio quadrado perfeito, utilizamos o método de completar quadrados.
	Ele consiste em realizar operações algébricas nessa equação de modo a transformá-la em um
	trinômio quadrado perfeito.
	
	\subsubsection{Completando quadrados}
	
	Para exemplificar, considere a equação:

	\[x^2 + 6x - 7 = 0\]

	Perceba que ela não pode ser simplificada em um trinômio quadrado perfeito, então realizaremos
	algumas operações algébricas para "completar" esse quadrado.

	\[x^2 + 6x = 7\]

	Como queremos que essa expressão seja da forma \(a^2 + 2\cdot ab + c^2\), podemos descobrir o 
	valor de $b$ realizando o cálculo \(\frac{6x}{2\cdot x} = 3\). A partir dele, verificamos que $b = 3$,
	e portanto devemos somar $b^2 = 9$ à nossa equação. Assim temos:

	\[x^2 + 6x + 9 = 7 + 9\]

	Fatorando o lado esquedo da equação, teremos:

	\[(x + 3)^2 = 16 \quad \leftrightarrow\]

	\[(x + 3)^2 = 4^2 \quad \leftrightarrow\]

	\[x + 3 = \pm 4 \quad \leftrightarrow\]

	\[x + 3 = 4 \quad \textrm{ou} \quad x + 3 = -4 \quad \leftrightarrow\]

	\[x = 1 \quad \textrm{ou} \quad x = -7\]

	E com isso, a equação foi solucionada por meio do método de
	completar quadrados.


	\subsection{Coeficiente "a" diferente de 1}

	Para as equações cujos coeficientes $a$ sejam diferentes de 1, basta
	realizar operações algébricas de modo a torná-lo 1. O jeito mais simples
	é dividir todos os termos da equação por $a$.

	\[ax^2 + bx + c = 0 \quad \leftrightarrow \]

	\[\frac{ax^2}{a} + \frac{bx}{a} + \frac{c}{a} = 0 \quad \leftrightarrow \]

	\[x^2 + \frac{bx}{a} + \frac{c}{a} = 0\]

	Com essa última equação, o método de completar quadrados pode ser aplicado como visto
	na seção anterior.

\end{document}
