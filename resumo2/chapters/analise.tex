\documentclass[../resumo.tex]{subfiles}
\graphicspath{{\subfix{../images/}}}

\begin{document}

	\subsection{Forma da função de terceiro grau}

	A equação de terceiro grau é da forma $a^3 + b^2 + c + d$, sendo $a$, $b$, $c$ e $d$ pertencentes ao 
	conjunto dos números reais, com a diferente de $0$.

	\subsubsection{Coeficiente a no gráfico da função de terceiro grau}

	O coeficiente dominante $a$ está diretamente relacionado ao comportamento da curva do 
	gráfico de uma função do terceiro grau, indicando o quadrante onde ela começa, e também, em
	qual deles ela termina.

	Em casos onde o coeficiente $a$ é maior que zero, temos um gráfico que começa no terceiro
	quadrante e termina no primeiro.

	\begin{figure}[H]
		\centering
		\includegraphics[width=.5\linewidth]{analise_casoal0}
		\caption{Caso $a > 0$}
		\label{fig:analise_casomenor0}
	\end{figure}

	Já para o caso onde $a$ é menor que zero, temos um gráfico que inicia no segundo quadrante 
	e termina no quarto.

	\begin{figure}[H]
		\centering
		\includegraphics[width=.5\linewidth]{analise_casoag0}
		\caption{Caso $a < 0$}
		\label{fig:analise_casomaior0}
	\end{figure}

	\subsubsection{Coeficiente b no gráfico da função de terceiro grau}

	O coeficiente $b$ determina o quão para esquerda o para direita ficará no gráfico, assim,
	quanto maior o $b$, mais deslocado à esquerda fica o gráfico, e quanto menor é o $b$, mais deslocado
	à direita fica o gráfico. Já para o caso de $b = 0$, temos uma simetria do gráfico

	\begin{figure}[H]
		\centering
		\includegraphics[width=1\linewidth]{analise_coeficiente}
		\caption{Efeito do coeficiente $b$ na função cúbica $f(x) = ax^3 + bx^2 + cx + d$}
		\label{fig:analise_coeficiente}
	\end{figure}

	\subsubsection{Coeficiente d no gráfico da função de terceiro grau}

	O coeficiente ponto de intersecção entre o gráfico e o eixo y.



	\subsubsection{Raízes da função de terceiro grau}

	Funções do terceiro grau possuem três raízes, que podem ser reais ou complexas. Essas
	raízes podem ser encontradas por fatoração.

	\begin{figure}[H]
		\centering
		\includegraphics[width=.5\linewidth]{analise_raizes}
		%\caption{}
		\label{fig:analise_raizes}
	\end{figure}

	Para achar as raízes de uma equação do 3º grau por fatoração, procure por raízes
	óbvias (ou testando valores de "$\frac{d}{a}$") para que possa dividir o polinômio e encontrar fatores, como
	usando a fatoração por agrupamento, ou a fatoração de cubos, ou ainda dividindo por um fator
	linear já conhecido da raiz. Se tiver uma raiz "$r$", poderá fatorar o polinômio da forma:
	$(x - r) (ax^2+ bx + c) = 0$. O produto de todos os fatores, $a(x - x')(x - x'')(x - x''')$, mostra as raízes $(x', x'', x''')$
	do polinômio.

	\subsubsection{Passo a passo para fatoração}

	\begin{itemize}
		\item Testar raízes racionais (se houver)
		\begin{itemize}
			\item Se a equação tiver raízes racionais, elas serão do tipo $\frac{p}{q}$, onde p é um divisor do
termo constante ($d$) e $q$ é um divisor do coeficiente líder ($a$)
			\item Teste os divisores de $d$ e $q$ no polinômio. Se um valor $x = r$ anular a equação, então $r$ é uma raiz
		\end{itemize}
		\item Fatorar o polinômio
		\begin{itemize}
			\item Se você encontrou uma raiz $r$, então $(x - r)$ é um fator do polinômio
			\item Divida o polinômio pelo fator $(x - r)$ para obter um polinômio do 2º grau. Você pode
fazer isso por divisão polinomial ou pelo método de dispositivo de Briot-Ruffini
		\end{itemize}
		\item Resolver o polinômio do 2º grau
		\begin{itemize}
			\item Encontre as raízes da equação do 2º grau resultante usando a fórmula de Bhaskara
		\end{itemize}
		\item As raízes são:
		\begin{itemize}
			\item As raízes são o valor $r$ que você já encontrou e as duas raízes do polinômio de 2º grau
		\end{itemize}
	\end{itemize}

	\textbf{Exemplo:}

	Para a função:

	\[ f(x) = x^3 - 6x^2 + 11x - 6 = 0 \]

	Teste raízes: Testando $x = 1$, obtemos:

	\begin{align*}
		1^3 - 6(1)^2 + 11(1) - 6 &= 0 \\
		1 - 6 + 11 - 6 &= 0 \\
	\end{align*}

	Logo, x=1 é uma raiz.

	Fatore, $(x - 1)$ é um fator. Dividindo o polinômio por $(x - 1)$ (usando Briot-Ruffini ou divisão), obtemos $x^2 - 5x + 6$.

	Resolva a equação quadrática, $x^2 - 5x + 6 = 0$. Usando Bhaskara, as raízes são $x = 2$ e $x = 3$.

	Raízes, As raízes da função são $x = 1$, $x = 2$ e $x = 3$.

	A forma fatorada da função seria:

	\[ (x - 1)\cdot(x - 2)\cdot(x - 3) = 0 \]

\end{document}
