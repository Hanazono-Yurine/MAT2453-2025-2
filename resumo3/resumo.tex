\documentclass[12pt, a4paper]{article}
\usepackage{graphicx} % Required for inserting images
\graphicspath{{images/}}
\usepackage[brazilian]{babel}
\usepackage[a4paper, left=30mm, right=20mm, top=30mm, bottom=20mm]{geometry}
\usepackage{float} % pra conseguir usar o \begin{figure}[H]
\usepackage{indentfirst} % pra fazer tab n 1 paragrafos dos capitulos
% package pra tabela
\usepackage{multirow}
\usepackage{array}
\usepackage{subfiles}
\usepackage{mathtools,amssymb}

\renewcommand{\contentsname}{Sumário}

\title{
	Resumo 3 de Cálculo I - MAT2453
}

\author{
	Alunos:
	\\  
	\\ {{PERSON-1}} - N.USP: {{PERSON-ID-1}}
	\\ {{PERSON-2}} - N.USP: {{PERSON-ID-2}}
	\\ {{PERSON-5}} - N.USP: {{PERSON-ID-5}}
	\\ {{PERSON-3}} - N.USP: {{PERSON-ID-3}}
	\\ {{PERSON-4}} - N.USP: {{PERSON-ID-4}}
	\\ {{PERSON-6}} - N.USP: {{PERSON-ID-6}}
	\\ \\
	Grupo: Filhos de Turing
	\\ Professor: Odilson Otavio Luciano
}

\date{\today}

\begin{document}

	\maketitle
	\thispagestyle{empty}
	\newpage

	\clearpage
	\thispagestyle{empty}
	\hfill
	\clearpage

	\tableofcontents
	
	\newpage

	\section{Introdução}

	\subfile{chapters/introducao}
	\newpage

	\section{Dedução}

	\subfile{chapters/deducao}
	\newpage

	\section{Xmax e Ymax – primeiro quadrante}

	\subfile{chapters/xymax}
	\newpage

	\begin{figure}[H]
			\centering
			\includegraphics[width=0.7\textwidth]{grafico}
			\caption{Gráfico para $a = 5$}
	\end{figure}

\end{document}
