\documentclass[../resumo.tex]{subfiles}
\graphicspath{{\subfix{../images/}}}

\begin{document}

	Dada a equação biquadrática $(x^2 + y^2)^2 = a^2(x^2 - y^2)$, desenvolvendo os dois lados,
	temos:

	\begin{align*}
		x^4 + 2x^2y^2 + y^4 &= a^2x^2 - a^2y^2 \quad &\longleftrightarrow \\
		y^4 + 2x^2y^2 + x^4 - a^2x^2 + a^2y^2 &= 0 \quad &\longleftrightarrow \\
		y^4 + (2x^2 + a^2)y^2 + (x^4 - a^2x^2) &= 0
	\end{align*}

	Ao isolar o $y^2$ , podemos chamá-lo de $\lambda$ assim, resultando em um
	polinômio de segundo grau em $\lambda$:

	\[\lambda^2 + (2x^2 + a^2)\lambda + (x^4 - a^2x^2) = 0\]

	Sendo assim, agora podemos aplicar a Fórmula de Bhaskara à função, considerando os coeficientes a' b' e c'
	abaixo,
	para encontrar o valor de $\lambda$:

	\begin{align*}
		a' &= 1 \\
		b' &= (2x^2 + a^2) \\
		c' &= (x^4 - a^2x^2) \\
		\lambda &= \frac{-(2x^2 + a^2) \pm \sqrt{a^2(8x^2 + a^2)}}{2} \\
		&= -(2x^2 + a^2) \pm a\sqrt{8x^2 + a^2}
	\end{align*}

	Assim, temos que:

	\[y = \pm \sqrt{\frac{-(2x^2 + a^2) \pm a\sqrt{(8x^2 + a^2)}}{2}}\]

	Ao analisar esse resultado, podemos concluir que

	\[a\sqrt{(8x^2 + a^2)} \geq 2x^2 + a^2\]

	Pois do contrário, $y$ seria um número complexo.

	Tendo em vista que

	\[a\sqrt{(8x^2 + a^2)} \geq 2x^2 + a^2\]

	Podemos desenvolver a inequação e concluir que:

	\begin{align*}
		a^2(8x^2 + a^2) &\geq 4x^4 + 4x^2a^2 + a^4 \quad &\longleftrightarrow \\
		4x^2a^2 - 4x^4 &\geq 0 \quad &\longleftrightarrow \\
		a^2 &\geq \frac{4x^4}{4x^2} \quad &\longleftrightarrow \\
		a^2 &\geq x^2 \quad &\longleftrightarrow \\
		|a| &\geq |x|
	\end{align*}

	Dessa forma, para $a > 0$, temos que $-a < x < a$, ou seja, a função só
	está definida entre $-a$ e $a$

\end{document}
