\documentclass[../resumo.tex]{subfiles}
\graphicspath{{\subfix{../images/}}}

\begin{document}

	A curva lemniscata é reconhecida por seu formato semelhante a um oito deitado. A lemniscata mais conhecida é
	a Lemniscata de Bernoulli, que é definida como o lugar geométrico dos pontos em um plano cujo produto
	das distâncias a dois pontos fixos, denominados focos, é constante e igual ao quadrado da metade da distância
	entre os focos.

	O estudo desta curva foi fundamental no desenvolvimento da teoria do cálculo, levando à importante
	descobertas sobre funções elípticas por diversos matemáticos, dentre eles, Carl Friedrich Gauss e 
	Leonhard Euler.

	Essa curva carrega consigo um grande simbolismo matemático e cultural, uma vez que ela é reconhecida como
	o símbolo do infinito, representando matematicamente o conceito abstrato de um número de dimensões
	imensuráveis, e culturalmente, os conceitos de eternidade, continuidade e ciclos sem fim. 

	A seguir, veremos a dedução da equação de uma curva lemniscata.

	\begin{figure}[H]
			\centering
			\includegraphics[width=0.7\textwidth]{lemniscate}
			\caption{Traço da Lemniscata de Bernoulli}
	\end{figure}

\end{document}
