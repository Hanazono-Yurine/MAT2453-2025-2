\documentclass[../resumo.tex]{subfiles}
\graphicspath{{\subfix{../images/}}}

\begin{document}

	Este trabalho acadêmico demonstra a viabilidade de usar funções de terceiro grau
	para projetar o perfil de lombadas com um foco em otimizar o conforto e a segurança
	viária.

	A modelagem matemática, baseada em cálculo diferencial, permite a criação de
	um perfil que garante uma transição suave para veículos que respeitam a velocidade de
	projeto, minimizando a aceleração vertical. Ao mesmo tempo, a natureza quadrática da
	relação entre a aceleração e a velocidade garante que a penalidade por excesso de velocidade
	seja significativa, incentivando a desaceleração.

	A aplicação das fórmulas em um exemplo prático valida o modelo e mostra como a engenharia
	pode usar princípios matemáticos avançados para resolver problemas de segurança pública
	de forma inovadora e eficaz.

\end{document}
