\documentclass[../resumo.tex]{subfiles}
\graphicspath{{\subfix{../images/}}}

\begin{document}
	
	A segurança viária é uma preocupação global, e as lombadas são dispositivos de desaceleração
	frequentemente usados para moderar a velocidade dos veículos.
	No entanto, o projeto de lombadas tradicionais muitas vezes compromete o conforto do motorista
	e dos passageiros, podendo até causar danos aos veículos se a velocidade de travessia não for adequada.
	A geometria da lombada é crucial para seu desempenho. Ao invés de usar perfis circulares ou trapezoidais simples,
	este trabalho propõe um modelo teórico de uma lombada com perfil otimizado, empregando funções de terceiro grau.
	O objetivo é projetar uma geometria que minimize a aceleração vertical para veículos que trafegam na velocidade
	máxima permitida, garantindo uma travessia suave e confortável, enquanto as acelerações em velocidades superiores
	atuam como um fator de penalidade, incentivando a conformidade com o limite de velocidade.

	O modelo é espelhado em relação ao seu ponto central, garantindo que o impacto e a aceleração sejam simétricos
	na entrada e na saída. A análise se baseia em noções de cálculo diferencial, usando a derivada para analisar a
	inclinação da tangente e a aceleração vertical do veículo.

\end{document}
