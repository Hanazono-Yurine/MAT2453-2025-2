\documentclass[../resumo.tex]{subfiles}
\graphicspath{{\subfix{../images/}}}

\begin{document}

	\subsection{Decomposição em frações parciais}

	Primeiro, fatoramos o denominador e decompomos em frações parciais

	\[ x^3 - x = x(x^2 - 1) = x(x - 1)(x + 1) \]

	Estabelecemos a decomposição:

	\begin{align*}
		\frac{x^2 + 1}{x(x-1)(x+1)} = \frac{A}{x} + \frac{B}{x - 1} + \frac{C}{x + 1}
	\end{align*}

	Multiplicando por $x(x-1)(x+1)$

	\[ x^2 + 1 = A(x-1)(x+1) + Bx(x+1) + ( x(x-1) ) \]

	\subsection{Cálculo dos coeficientes}

	\textbf{Para $x = 0$}

	\[ 0^2 + 1 = A(0 - 1)(0 + 1) \Rightarrow 1 = -A \Rightarrow A = -1\]

	\textbf{Para $x = 1$}

	\[ 1^2 + 1 = B(1)(1 + 1) \Rightarrow 2 = 2B \Rightarrow B = 1\]

	\textbf{Para $x = -1$}

	\[ (-1)^2 + 1 = C(-1)(-1 -1) \Rightarrow 2 = 2C \Rightarrow C = 1\]

	A função reescrita em trmos de potencias negativas é:

	\[ f(x) = \frac{-1}{x} + \frac{1}{x-1} + \frac{1}{x+1} \]

	\[ f(x) = (-x)^{-1} + (x - 1)^{-1} + (x + 1)^{-1} \]

	\subsection{Fórmula geral para a n-éssima derivada}

	Para uma função de forma $g(x) = (x + c)^{-1}$, a n-éssima derivada é dada por:

	\[ \frac{d^n}{dx^n} (x + c)^{-1} = (-1)^n n!(x = c)^{-(n-1)} \]

	Aplicando essa fórmula para a n-éssima derivada de $f(x)$

	\[ f^n(x) = \frac{d^n}{dx^n}(-x^{-1}) + \frac{d^n}{dx^n}((x-1)^{-1}) + \frac{d^n}{dx^n}((x-1)^{-1}) \]
	\[ f^n(x) = -[(-1)^n n!x^{-(n-1)}] + [(-1)^n n!(x-1)^{-(n+1)}] + [(-1)^n n! (x+1)^{-(n+1)}] \]

	Fatorando o termo comum $(-1)^nn!$:

	\[ f^n(x) = (-1)^n n! [-x^{-(n+1)} + (x-1)^{-(n+1)} + (x+1)^{-(n+1)}] \]

	\subsection{A centésima derivada $(n=100)$}

	\begin{align*}
		(-1)^n &= (-1)^{100} = 1 \\
		n! &= 100! \\
		n + 1 &= 101 \\
		f^{100}(x) &= (-1)^{100} \cdot 100! [-x^{-(100+1)} + (x-1)^{-(100+1)} + (x+1)^{-(100+1)}] \\
		f^{100}(x) &= 100! [-x^{-101} + (x-1)^{-101} + (x+1)^{-101}]
	\end{align*}

	Reorganizando e usando a notação de frações:

	\[ f^{100}(x) = 100! \left( \frac{1}{(x-1)^{101}} + \frac{1}{(x+1)^{101}} - \frac{1}{(x^{101})} \right) \]

\end{document}
