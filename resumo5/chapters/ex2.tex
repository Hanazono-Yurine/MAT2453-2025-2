O problema consiste em determinar o caminho mais
curto do ponto Z ao ponto W, tocando uma linha reta (segmento UV) em um ponto P.
Por definição, o caminho mais curto entre esses dois pontos é uma reta. Sendo assim, para que
a distância $ZP + PW$ seja mínima, Z, P e W devem ser colineares.

\subsection{Nomenclatura e definição geométrica}
Segundo o diagrama no quadro, temos os seguintes valores:

\begin{itemize}
    \item q: a distância perpendicular do ponto Z à linha UV.
    \item p: a distância perpendicular do ponto W à linha UV.
    \item d: a distância horizontal entre as projeções dos pontos Z e W na linha UV.
    \item x: a distância da projeção de Z até o ponto P.
\end{itemize}

Quando Z, P e W formam uma linha reta, nota-se que são formados dois triângulos retângulos
semelhantes, como o diagrama apresentado em aula mostra.

Esses dois triângulos retângulos (um com catetos q e x, e outro com catetos p e $d - x$)
são semelhantes.Isso se dá pois eles compartilham os mesmos ângulos (opostos ao vértice P),
e um ângulo reto.

\subsection{Utilizando a semelhança de triângulos}

Pela semelhança de triângulos, a razão entre os lados correspondentes é igual.
Portanto, a razão entre os catetos verticais e os catetos horizontais de ambos os
triângulos deve ser a mesma. Logo, temos que:

\[\frac{q}{x} = \frac{p}{d - x}\]

\subsection{Resolvendo para x}
A partir da última equação, podemos encontrar o valor de x que minimiza a distância:

\begin{align*}
    q(d - x) &= px \\
    qd - qx &= px \\
    qd &= px + qx \\
    qd &= x(p + q)
\end{align*}

Portanto, conclui-se que o ponto P que minimiza a função
$f(x) = ZP + PW$ é aquele cuja distância x da extremidade esquerda U é dada por:
\[x = \frac{qd}{p + q}\]
