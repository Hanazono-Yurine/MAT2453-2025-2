\documentclass[../resumo.tex]{subfiles}
\graphicspath{{\subfix{../images/}}}

\begin{document}

\subsection{Comprimento máximo de um segmento passando pela esquina}

	Considere dois corredores perpendiculares com larguras $L$ (horizontal) e $l$ (vertical).
	Queremos o maior comprimento $d$ de um segmento rígido que pode ser transportado ao longo da
	esquina sem dobrar. Coloque um sistema de coordenadas com o canto interior da esquina na origem;
	o corredor horizontal se estende para $x \geq 0$ com $0 \leq y \leq l$, e o corredor vertical
	para $y \geq 0$ com $0 \leq x \leq L$.

	\subsection{Modelo geométrico}

	Posicione o segmento de modo que, ao atravessar a esquina, uma extremidade esteja no corredor
	horizontal e a outra no corredor vertical. Imagine o segmento tocando as paredes internas em dois pontos:
	um contato com a parede superior do corredor horizontal (reta $y = l$) e outro contacto com a parede direita
	do corredor vertical (reta $x = L$). Seja $\theta$ o ângulo que o segmento faz com o eixo horizontal
	quando parte do trecho horizontal; representamos por a a projeção horizontal do segmento entre o ponto onde
	toca $y = l$ e o canto, e por $b$ a projeção vertical entre o ponto onde toca $x = L$ e o canto.
	Assim temos que o comprimento do segmento deve satisfazer
	
	\[ d = a + b \]
	
	quando projetado nas direções dos corredores, porém as projeções $a$ e $b$ relacionam-se com o comprimento real
	pela geometria do ângulo. Para cada posição $\theta$, as projeções mínimas requeridas para que o segmento passe são

	\[ a = \frac{l}{\sin\theta},\qquad b = \frac{L}{\cos\theta},\]

	pois, ao projetar a porção que fica no corredor horizontal sobre a direção $x$, a distância ao longo do segmento que
	corresponde à altura $l$ é $\dfrac{l}{sin\theta}$; analogamente para $b$ com $\dfrac{L}{cos\theta}$.
	Portanto, o comprimento necessário como função de $\theta$ é

	\[ d(\theta)=\frac{l}{\sin\theta}+\frac{L}{\cos\theta},\qquad \theta\in(0,\tfrac{\pi}{2}) \]

	\subsection{Otimização}

	Queremos o mínimo possível de $d(\theta)$ (esse mínimo é o comprimento máximo que ainda cabe; para comprimentos maiores
	não é possível). Derivando em relação a $\theta$ e anulando a derivada,
	
	\[ \frac{d}{d\theta}d(\theta) = -l\frac{\cos\theta}{\sin^2\theta}+L\frac{\sin\theta}{\cos^2\theta}=0 \]
	
	Multiplicando ambos os lados por \(\sin^2\theta\cos^2\theta\) obtemos
	
	\begin{align*}
		- l\cos^3\theta + L\sin^3\theta = 0
		\quad\Longrightarrow\quad
		L\sin^3\theta = l\cos^3\theta.
	\end{align*}
	
	Isto dá a relação

	\begin{align*}
		\left(\frac{\sin\theta}{\cos\theta}\right)^3 = \frac{l}{L}
		\quad\Longrightarrow\quad
		\tan\theta = \left(\frac{l}{L}\right)^{1/3}.
	\end{align*}

	Substitua essa condição em $d(\theta)$. É conveniente escrever \(\sin\theta\) e \(\cos\theta\) em termos de $t = \tan\theta$:

	\[
	\sin\theta=\frac{t}{\sqrt{1+t^2}},\qquad
	\cos\theta=\frac{1}{\sqrt{1+t^2}}.
	\]

	Com \(t=(l/L)^{1/3}\) obtemos

	\[
	\frac{1}{\sin\theta}=\frac{\sqrt{1+t^2}}{t},\qquad
	\frac{1}{\cos\theta}=\sqrt{1+t^2}.
	\]

	Logo

	\[
	d_{\min}=l\frac{\sqrt{1+t^2}}{t}+L\sqrt{1+t^2}=(l/t+L)\sqrt{1+t^2}.
	\]

	Substituindo t e simplificando:
	\[
	l/t = l\,(l/L)^{-1/3}=l^{2/3}L^{1/3},\qquad
	L = L^{2/3}L^{1/3}=L^{2/3}L^{1/3}.
	\]

	Assim

	\[
	l/t+L = l^{2/3}L^{1/3}+L^{2/3}L^{1/3}=(l^{2/3}+L^{2/3})\,L^{1/3}.
	\]

	Também

	\[
	\sqrt{1+t^2}=\sqrt{1+(l/L)^{2/3}}=L^{-1/3}\sqrt{L^{2/3}+l^{2/3}}.
	\]

	Multiplicando os fatores:

	\begin{align*}
		d_{\min} &= (l^{2/3}+L^{2/3})\, L^{1/3}\times L^{-1/3}\sqrt{L^{2/3}+l^{2/3}} \\
						 &= (l^{2/3}+L^{2/3})\sqrt{l^{2/3}+L^{2/3}}
	\end{align*}

	Finalmente,

	\[
		d_{\max}=\bigl(l^{2/3}+L^{2/3}\bigr)^{3/2}
	\]

	que é a expressão desejada.

	\subsection{Observação curta}

	A minimização de $\dfrac{l}{\sin\theta}+\dfrac{L}{\cos\theta}$ conduz à condição $\tan^3\theta=\dfrac{l}{L}$ e da simplificação algébrica
	surge a expressão em potências $\dfrac{2}{3}$ e $\dfrac{3}{2}$ acima.

\end{document}
